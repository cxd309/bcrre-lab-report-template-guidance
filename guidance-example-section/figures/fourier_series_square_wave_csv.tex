\begin{figure}[H]
    \centering
    \begin{tikzpicture}
        \begin{axis}[
            width=0.8\textwidth,
            height=0.6\textwidth,
            axis lines = left,
            xlabel = {x},
            ylabel = {y},
            xmajorgrids=true,
            ymajorgrids=true,
            ymin=-1,
            ymax=1,
            xmin=0,
            xmax=7
        ]
            \addplot[color=blue] table [x=x, y=1_term, col sep=comma]{guidance-example-section/data/fourier_series_square_wave.csv};
            \addlegendentry{1 term}
            \addplot[color=red] table [x=x, y=2_term, col sep=comma]{guidance-example-section/data/fourier_series_square_wave.csv};
            \addlegendentry{2 terms}
            \addplot[color=brown] table [x=x, y=3_term, col sep=comma]{guidance-example-section/data/fourier_series_square_wave.csv};
            \addlegendentry{3 terms}
            \addplot[color=black] table [x=x, y=20_term, col sep=comma]{guidance-example-section/data/fourier_series_square_wave.csv};
            \addlegendentry{20 terms}
        \end{axis}
    \end{tikzpicture}
    \caption{Fourier series approximation of a square wave with varying number of cosine terms. Plotted with pgfplots from stored csv data}
    \label{fig:fourier_series_square_wave_csv}
\end{figure}